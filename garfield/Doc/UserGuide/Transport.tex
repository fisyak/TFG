On a phenomenological level, 
the drift of charge carriers under the influence of an 
electric field \(\mathbf{E}\) and a magnetic field \(\mathbf{B}\) 
is described by the first order equation of motion
\begin{equation}\label{Eqn:FirstOrderEquationOfMotion}
  \dot{\mathbf{r}} = 
  \mathbf{v}_{\text{d}}\bigl(\mathbf{E}\left(\mathbf{r}\right), 
                             \mathbf{B}\left(\mathbf{r}\right)\bigr),
\end{equation}
where \(\mathbf{v}_{\text{d}}\) is the drift velocity. 
For the solution of \eqref{Eqn:FirstOrderEquationOfMotion}, 
two methods are available in Garfield++:
\begin{itemize}
  \item
  Runge-Kutta-Fehlberg integration, and
  \item
  Monte Carlo integration (\texttt{AvalancheMC}).
\end{itemize}

For accurate simulations of electron trajectories 
in small-scale structures 
(with characteristic dimensions comparable to the electron mean free path),
and also for detailed calculations of ionisation and excitation processes, 
transporting electrons on a microscopic level -- 
i.~e. based on the second-order equation of motion --
is the method of choice. 
Microscopic tracking of electrons is dealt with by the class 
\texttt{AvalancheMicroscopic}.  
\section{Runge-Kutta-Fehlberg Integration}
This method is implemented in the class \texttt{DriftLineRKF}.

\section{Monte Carlo Integration}
In the class \texttt{AvalancheMC}, Eq.~\eqref{Eqn:FirstOrderEquationOfMotion}
is integrated in a stochastic manner:
\begin{itemize}
  \item
  a step of length \(\Delta{s} = v_{\text{d}}\Delta{t}\) 
  in the direction of the
  drift velocity \(\mathbf{v}_{\text{d}}\) 
  at the local field is calculated (with either the 
  time step \(\Delta{t}\) or the distance \(\Delta{s}\) 
  being specified by the user);
  \item
   a random diffusion step
   is sampled from three uncorrelated Gaussian distributions
   with standard deviation \(\sigma_{L} = D_{L}\sqrt{\Delta{s}}\)
   for the component parallel to the drift velocity and
   standard deviation
   \(\sigma_{T} = D_{T}\sqrt{\Delta{s}}\) for the two
   transverse components;
   \item
   the two steps are added vectorially and the location is updated.
\end{itemize}
The functions for setting the step size are 
\begin{lstlisting}
void SetTimeSteps(const double d = 0.02);
void SetDistanceSteps(const double d = 0.001);
void SetCollisionSteps(const int n = 100);
\end{lstlisting} 
In the first case the integration is done 
using fixed time steps (default: 20 ps), 
in the second case using fixed distance steps (default: 10 \(\mu\)m). 
Calling the third function instructs the class to 
do the integration with exponentially distributed time steps 
with a mean equal to the specified multiple of the ``collision time'' 
\begin{equation*}
  \tau = \frac{m v_{d}}{q E}.
\end{equation*}
The third method is activated by default.

Drift line calculations are started using 
\begin{lstlisting}
bool DriftElectron(const double x0, const double y0, const double z0,
                   const double t0);
bool DriftHole(const double x0, const double y0, const double z0,
               const double t0);
bool DriftIon(const double x0, const double y0, const double z0,
              const double t0);
\end{lstlisting}
\begin{description}
  \item[x0, y0, z0, t0] initial position and time
\end{description}
The trajectory can be retrieved using
\begin{lstlisting} 
int GetNumberOfDriftLinePoints() const;
void GetDriftLinePoint(const int i,
                       double& x, double& y, double& z, double& t);
\end{lstlisting}

The calculation of an avalanche initiated by an electron, 
a hole or an electron-hole pair is done using
\begin{lstlisting}
bool AvalancheElectron(const double x0, const double y0, const double z0,
                       const double t0, const bool hole = false);
bool AvalancheHole(const double x0, const double y0, const double z0,
                   const double t0, const bool electron = false);
bool AvalancheElectronHole(const double x0, const double y0, const double z0,
                           const double t0);
\end{lstlisting}
The flags \texttt{hole} and \texttt{electron} specify whether 
multiplication of holes and electrons, respectively, should be 
taken into account in the simulation. 
In case of gas-based detectors, only \texttt{AvalancheElectron} with 
\texttt{hole = false} is meaningful. 

The starting and endpoints of electrons in the avalanche can be 
retrieved using
\begin{lstlisting}
int GetNumberOfElectronEndpoints();
void GetElectronEndpoint(const int i,
                         double& x0, double& y0, double& z0, double& t0,
                         double& x1, double& y1, double& z1, double& t1,
                         int& status) const;
\end{lstlisting}
\begin{description}
  \item[i] index of the electron
  \item[x0, y0, z0, t0] initial position and time of the electron
  \item[x1, y1, z1, t1] final position and time of the electron
  \item[status] status code indicating why the tracking of the electron was stopped.  
\end{description}
Analogous functions are available for holes and ions.

The functions
\begin{lstlisting}
void EnableMagneticField();
\end{lstlisting}
instructs the class to consider not only the electric but also the magnetic field
in the evaluation of the transport parameters.  
By default, magnetic fields are not taken into account.

For debugging purposes, attachment and diffusion can be switched off using
\begin{lstlisting}
void DisableAttachment();
void DisableDiffusion();
\end{lstlisting}

A time interval can be set using
\begin{lstlisting}
void SetTimeWindow(const double t0, const double t1);
\end{lstlisting}
\begin{description}
  \item[t0] lower limit of the time window
  \item[t1] upper limit of the time window 
\end{description}
Only charge carriers with a time coordinate \(t \in \left[t_{0}, t_{1}\right]\) 
are tracked. If the time coordinate of a particle crosses the upper limit, 
it is stopped and assigned the status code -17.
Slicing the calculation into time steps can be useful for instance 
for making a movie of the avalanche evolution 
or for calculations involving space charge.
The time window can be removed using
\begin{lstlisting}
void UnsetTimeWindow();
\end{lstlisting}
 
\section{Microscopic Tracking}\label{Sec:MicroscopicTracking}
Microscopic tracking is (at present) only possible for electrons. 
It is implemented in the class \texttt{AvalancheMicroscopic}. 
A calculation is started by means of
\begin{lstlisting}
void AvalancheElectron(const double x0, const double y0, const double z0,
                       const double t0, const double e0,
                       const double dx0 = 0., const double dy0 = 0., const double dz0 = 0.);
\end{lstlisting}
\begin{description}
  \item[x0, y0, z0, t0] initial position and time
  \item[e0] initial energy (eV)
  \item[dx0, dy0, dz0] initial direction 
\end{description}
If the norm of the direction vector is zero, 
the initial direction is randomized.

After the calculation is finished, the number of electrons 
(\texttt{ne}) and ions (\texttt{ni})  
produced in the avalanche can be retrieved using
\begin{lstlisting}
void GetAvalancheSize(int& ne, int& ni);
\end{lstlisting}
Information about the ``history'' of each avalanche electron can be 
retrieved by
\begin{lstlisting}
int GetNumberOfElectronEndpoints();
void GetElectronEndpoint(const int i, 
               double& x0, double& y0, double& z0, double& t0, double& e0,
               double& x1, double& y1, double& z1, double& t1, double& e1,
                 int& status); 
\end{lstlisting}
\begin{description}
  \item[i] index of the electron
  \item[x0, y0, z0, t0, e0] initial position, time and energy of the electron
  \item[x1, y1, z1, t1, e1] final position, time and energy of the electron
  \item[status] status code indicating why the tracking of the electron was stopped.  
\end{description}
A list of status codes is given in Table~\ref{Tab:DriftLineStatusCodes}.

The function
\begin{lstlisting}
bool DriftElectron(const double x0, const double y0, const double z0,
                   const double t0, const double e0,
                   const double dx0, const double dy0, const double dz0);
\end{lstlisting}
traces only the initial electron but not the secondaries 
produced along its drift path 
(the input parameters are the same as for \texttt{AvalancheElectron}).

\begin{table}
  \centering
  \begin{tabular}{r l}
    \toprule
    status code &  meaning\\
    \midrule
     -1 & particle left the drift area        \\
     -3 & calculation abandoned (error, should not happen) \\
     -5 & particle not inside a drift medium  \\
     -7 & attachment                          \\
    -16 & energy below transport cut          \\
    -17 & outside the time window             \\
    \bottomrule
  \end{tabular}
  \caption{Status codes for the termination of drift lines.}
  \label{Tab:DriftLineStatusCodes}
\end{table}

The electron energy distribution can be extracted in the following way:
\begin{lstlisting}
AvalancheMicroscopic* aval = new AvalancheMicroscopic();
// Make a histogram (100 bins between 0 and 100 eV).
TH1F* hEnergy = new TH1F("hEnergy", "Electron energy", 100, 0., 100.);
// Pass the histogram to the avalanche class.
aval->EnableElectronEnergyHistogramming(hEnergy);
\end{lstlisting} 
After each collision, 
the histogram is filled with the current electron energy. 

The effect of magnetic fields can be included 
in the stepping algorithm using the function
\begin{lstlisting}
void EnableMagneticField();
\end{lstlisting}
By default, magnetic fields are not taken into account in the calculation.

Using 
\begin{lstlisting}
void EnableAvalancheSizeLimit(const int size);
\end{lstlisting}
the size of an electron avalanche can be limited. 
After the avalanche has reached the specified max. size, 
no further secondaries are added to the stack of electrons to be transported.  

Like in \texttt{AvalancheMC} a time window can be set/unset using
\begin{lstlisting}
void SetTimeWindow(const double t0, const double t1);
void UnsetTimeWindow();
\end{lstlisting}

An energy threshold for transporting electrons can be applied using 
\begin{lstlisting}
void SetElectronTransportCut(const double cut);
\end{lstlisting}
\begin{description}
  \item[cut] energy threshold (in eV)
\end{description}
The tracking of an electron is aborted if its energy falls below the 
transport cut. This option can be useful for \(\delta\) electron studies in 
order to stop the calculation once the energy of an electron 
is below the ionization potential of the gas. 
The transport cut can be removed by setting the threshold to a negative value.
By default, no cut is applied.

In order to extract information from the avalanche on a collision-by-collision basis, 
so-called ``user handles'' are available. 
\begin{lstlisting}
 void SetUserHandleStep(void (*f)(double x, double y, double z,
                                  double t, double e,
                                  double dx, double dy, double dz,
                                  bool hole));
void UnsetUserHandleStep();
void SetUserHandleAttachment(void (*f)(double x, double y, double z,
                                       double t,
                                       int type, int level, Medium* m));
void UnsetUserHandleAttachment();
void SetUserHandleInelastic(void (*f)(double x, double y, double z,
                                      double t,
                                      int type, int level, Medium* m));
void UnsetUserHandleInelastic();
void SetUserHandleIonisation(void (*f)(double x, double y, double z,
                                       double t,
                                       int type, int level, Medium* m));
void UnsetUserHandleIonisation();
\end{lstlisting}
The function specified in \texttt{SetUserHandleStep} is called 
prior to each free-flight step. 
The parameters passed to this function are 
\begin{description}
  \item[x, y, z, t] 
  position and time, 
  \item[e]
  energy before the step
  \item[dx, dy, dz] 
  direction,
  \item[hole]
  flag indicating whether the particle is an electron or a hole.
\end{description}  
The ``user handle'' functions for attachment, ionisation, and inelastic collisions 
are called each time a collision of the respective type occurs.  
In this context, inelastic collisions also include excitations. 
The parameters passed to these functions are 
\begin{description}
  \item[x, y, z, t]
  the location and time of the collision, 
  \item[type]
  the type of collision (see Table~\ref{Tab:ElectronCollisionType}), 
  \item[level]
  the index of the cross-section term (as obtained from the \texttt{Medium}),
  \item[m]
   a pointer to the current \texttt{Medium}. 
\end{description}
In the following example we want 
all excitations which occur to undergo a special treatment. 
\begin{lstlisting}
void userHandle(double x, double y, double z, double t,
                int type, int level, Medium* m) {

  // Check if the collision is an excitation.
  if (type != 4) return;
  // Do something (e.~g. fill a histogram, simulate the emission of a VUV photon) 
  ...
} 

int main(int argc, char* argv[]) {

  // Setup gas, geometry, and field
  ...
  AvalancheMicroscopic* aval = new AvalancheMicroscopic();
  ...
  aval->SetUserHandleInelastic(userHandle);
  double x0 = 0., y0 = 0., z0 = 0., t0 = 0.;
  double e0 = 1.;
  aval->AvalancheElectron(x0, y0, z0, t0, e0, 0., 0., 0.); 
  ...

}

\end{lstlisting}
 
\section{Visualizing Drift Lines}

For plotting drift lines and tracks the class \texttt{ViewDrift} can be used. 
After attaching a \texttt{ViewDrift} object to a transport class, e.~g. using
\begin{lstlisting}
void AvalancheMicroscopic::EnablePlotting(ViewDrift* view);
void AvalancheMC::EnablePlotting(ViewDrift* view);
void Track::EnablePlotting(ViewDrift* view);
\end{lstlisting}
\texttt{ViewDrift} stores the trajectories which are calculated by the 
transport class. 
The drawing of the trajectories is triggered by the function
\begin{lstlisting}
void ViewDrift::Plot();
\end{lstlisting}

In case of \texttt{AvalancheMicroscopic}, it is usually not advisable to 
plot every single collision. The number of collisions 
to be skipped for plotting can be set using
\begin{lstlisting}
void AvalancheMicroscopic::SetCollisionSteps(const int n);
\end{lstlisting}
\begin{description}
  \item[n] number of collisions to be skipped
\end{description}
Note that this setting does not affect the transport of the electron as such, 
the electron is always tracked rigorously through single collisions.
