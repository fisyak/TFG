Signals are calculated using the Shockley-Ramo theorem. 
The current \(i\left(t\right)\) induced by a particle with charge 
\(q\) at a position \(\mathbf{r}\) moving at a velocity \(\mathbf{v}\)
is given by
\begin{equation*}
  i\left(t\right) = -q \mathbf{v} \cdot \mathbf{E}_{w}\left(\mathbf{r}\right),
\end{equation*}
where \(\mathbf{E}_{w}\) is the so-called weighting field for the 
electrode to be read out. 

The basic steps for calculating the current induced 
by the drift of electrons and ions/holes are:
\begin{enumerate}
  \item
  Prepare the weighting field for the electrode to be read out. 
  This step depends on the field calculation technique 
  (i.~e. the type of \texttt{Component}) which is used 
  (see Chapter~\ref{Chap:Components}). 
  \item
  Tell the \texttt{Sensor} that you want to use this 
  weighting field for the signal calculation. 
  \begin{lstlisting}
void Sensor::AddElectrode(ComponentBase* cmp, std::string label);
  \end{lstlisting}
  where \texttt{cmp} is a pointer to the \texttt{Component} 
  which calculates the weighting field, and \texttt{label} 
  (in our example \texttt{"readout"}) is 
  the name you have assigned to the weighting field in the previous step.
  \item
  Setup the binning for the signal calculation.
  \begin{lstlisting}
void Sensor::SetTimeWindow(const double tmin, const double tstep, 
                           const int nbins);
  \end{lstlisting}
  The first parameter in this function is the lower time limit (in ns), 
  the second one is the bin width (in ns), and the last one 
  is the number of time bins.
  \item
  Switch on signal calculation in the transport classes using 
  \begin{lstlisting}
void AvalancheMicroscopic::EnableSignalCalculation();
void AvalancheMC::EnableSignalCalculation();
  \end{lstlisting}
  The \texttt{Sensor} then records and accumulates the signals of all 
  avalanches and drift lines which are simulated.
  \item
  The calculated signal can be retrieved using 
  \begin{lstlisting}
double Sensor::GetSignal(const std::string label, const int bin);
double Sensor::GetElectronSignal(const std::string label, const int bin);
double Sensor::GetIonSignal(const std::string label, const int bin); 
  \end{lstlisting}
  The functions \texttt{GetElectronSignal} and 
  \texttt{GetIonSignal} return the signal induced by negative 
  and positive charges, respectively. \texttt{GetSignal} returns 
  the sum of both electron and hole signals.   
  \item
  After the signal of a given track is finished, call
  \begin{lstlisting}
void Sensor::ClearSignal();
  \end{lstlisting}
  to reset the signal to zero.
\end{enumerate}

For plotting the signal, the class \texttt{ViewSignal} can be used.
As an illustration of the above recipe consider the following example. 
\begin{lstlisting}
// Electrode label
const std::string label = "readout";
// Setup the weighting field.
// In this example we use a FEM field map.
ComponentAnsys123* fm = new ComponentAnsys123();
...
fm->SetWeightingField("WPOT.lis", label);

Sensor* sensor = new Sensor();
sensor->AddComponent(fm);
sensor->AddElectrode(fm, label);
// Setup the binning (0 to 100 ns in 100 steps).
const double tStart =   0.;
const double tStop  = 100.;
const int nSteps = 100;
const double tStep = (tStop - tStart) / nSteps;

AvalancheMicroscopic* aval = new AvalancheMicroscopic();
aval->SetSensor(sensor);
aval->EnableSignalCalculation();
// Calculate some drift lines.
...
// Plot the induced current.
ViewSignal* signalView = new ViewSignal(tStart, tStep, nSteps);
signalView->SetSensor(sensor);
signalView->Plot(label);
\end{lstlisting}

\section{Readout Electronics}

In order to model the signal-processing by the front-end electronics, the 
``raw signal'' -- i.~e. the induced current -- 
can be convoluted with a so-called ``transfer function''. 
The transfer function to be applied can be set using
\begin{lstlisting}
void Sensor::SetTransferFunction(double (*f)(double t));
\end{lstlisting}
where \texttt{double f(double t)} is a function provided by the user, 
or using
\begin{lstlisting}
void Sensor::SetTransferFunction(std::vector<double> times,
                                 std::vector<double> values);
\end{lstlisting}
in which case the transfer function will be calculated by 
interpolation of the values provided in the table.

In order to convolute the presently stored signal with the 
transfer function (specified using the above function), 
the function
\begin{lstlisting}
bool Sensor::ConvoluteSignal();
\end{lstlisting}
can be called. 

As an example, consider the following transfer function
\begin{equation*}
  f\left(t\right) = e\frac{t}{\tau}\text{e}^{1 - t/\tau}, \qquad
  \tau = 25~\text{ns}
\end{equation*}

\begin{lstlisting}
double transfer(double t) {

  const double tau = 25.;
  return (t / tau) * exp(1 - t / tau);

}

int main(int argc, char* argv[]) {

  // Setup component, media, etc.
  ...
  Sensor* sensor = new Sensor();
  sensor->SetTransferFunction(transfer);
  // Calculate the induced current.
  ...
  // Apply the transfer function.
  sensor->ConvoluteSignal();
  ...
}
 
\end{lstlisting}
