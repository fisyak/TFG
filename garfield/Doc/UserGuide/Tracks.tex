The purpose of classes of the type \texttt{Track} is to simulate  
ionization patterns produced by fast charged particles traversing the detector. 

The type of the primary particle is set by the function
\begin{lstlisting}
void SetParticle(std::string particle);
\end{lstlisting}
\begin{description}
  \item[particle]
  name of the particle 
\end{description}
Only particles which are sufficiently long lived to leave a 
track in a detector are considered.
A list of the available particles is given 
in Table~\ref{Tab:ParticleNames}.

The kinematics of the charged particle can be defined 
by means of a number of equivalent methods:
  \begin{itemize}
    \item
    the total energy (in eV) of the particle,
    \item
    the kinetic energy (in eV) of the particle,
    \item
    the momentum (in eV/\(c\)) of the particle,
    \item
    the (dimension-less) velocity \(\beta = v / c\), 
    the Lorentz factor \(\gamma = 1 / \sqrt{1 - \beta^2}\) 
    or the product \(\beta\gamma\) of these two variables.
  \end{itemize}
The corresponding functions are
\begin{lstlisting}
void SetEnergy(const double e);
void SetKineticEnergy(const double ekin);
void SetMomentum(const double p);
void SetBeta(const double beta);
void SetGamma(const double gamma);
void SetBetaGamma(const double bg);
\end{lstlisting}

\begin{table}
  \centering
  \begin{tabular}{l l . r}
    \toprule
    particle & & \multicolumn{1}{c}{mass [MeV/\(c^{2}\)]} & charge \\
    \midrule
    \(e\)     & \texttt{electron, e-} & 0.510998910 & \(-1\) \\
    \(e^{+}\) & \texttt{positron, e+} & 0.510998910 & \(+1\) \\
    \(\mu^{-}\) & \texttt{muon, mu-}  & 105.658367  & \(-1\) \\
    \(\mu^{+}\) & \texttt{mu+}        & 105.658367  & \(+1\) \\
    \(\pi^{-}\) & \texttt{pion, pi, pi-} & 139.57018 & \(-1\) \\
    \(\pi^{+}\) & \texttt{pi+}           & 139.57018 & \(+1\) \\
    \(K^{-}\)   & \texttt{kaon, K, K-}   & 493.677   & \(-1\) \\
    \(K^{+}\)   & \texttt{K+}            & 493.677   & \(+1\) \\
    \(p\)       & \texttt{proton, p}     & 938.272013 & \(+1\) \\
    \(\overline{p}\) & \texttt{anti-proton, antiproton, p-bar} & 938.272013 & \(-1\) \\
    \(d\)       & \texttt{deuteron, d} & 1875.612793 & \(+1\) \\
  \bottomrule
  \end{tabular}
  \caption{Available charged particles.}
  \label{Tab:ParticleNames}
\end{table}

A track is initialized by means of
\begin{lstlisting}
void NewTrack(const double x0, const double y0, const double z0,
               const double t0,
               const double dx0, const double dy0, const double dz0);
\end{lstlisting}
\begin{description}
  \item[x0, y0, z0] initial position (in cm)
  \item[t0] starting time
  \item[dx0, dy0, dz0] initial direction vector
\end{description}
The starting point of the track has to be inside an ionizable medium. 
Depending on the type of \texttt{Track} class, there can be 
further restrictions on the type of \texttt{Medium}.
If the specified direction vector has zero norm, an isotropic random vector 
will be generated.
 
After successful initialization, the ``clusters'' produced along the track
can be retrieved by
\begin{lstlisting}
bool GetCluster(double& xcls, double& ycls, double& zcls, double& tlcs,
                int& n, double& e, double& extra);
\end{lstlisting}
\begin{description}
  \item[xcls, ycls, zcls, tcls] 
  position (and time) of the ionizing collision
  \item[n] number of electrons produced in the collision
  \item[e] transferred energy (in eV)
\end{description}
The function returns \texttt{false} if the list of clusters is exhausted
 or if there is no valid track.

The concept of a ``cluster'' deserves some explanation. 
In the present context it refers to the energy loss in a single ionizing 
collision of the primary charged particle and the secondary 
electrons produced in this process. 


\section{Heed}

The program Heed \cite{Smirnov2005} is an implementation 
of the photo-absorption ionization (PAI) model. 
It was written by I. Smirnov.
An interface to Heed is available through the class \texttt{TrackHeed}. 

After calling \texttt{GetCluster}, 
one can retrieve details about the 
electrons in the present cluster using
\begin{lstlisting}
bool GetElectron(const unsigned int i, double& x, double& y, double& z,
                 double& t, double& e, 
                 double& dx, double& dy, double& dz);
\end{lstlisting}

\subsection{Delta Electron Transport}

Heed simulates the energy degradation of \(\delta\) electrons and 
the production of secondary (``conduction'') electrons 
using a phenomenological algorithm described in Ref.~\cite{Smirnov2005}.

The asymptotic \(W\) value (eV) and the Fano factor of a 
\texttt{Medium} can be specified by the user by means of the functions
\begin{lstlisting}
void Medium::SetW(const double w);
void Medium::SetFanoFactor(const double f);
\end{lstlisting}
If these parameters are not set, Heed uses internal default values. 
The default value for the Fano factor is \(F = 0.19\).

The transport of \(\delta\) electrons can be activated or deactivated 
using
\begin{lstlisting}
void EnableDeltaElectronTransport();
void DisableDeltaElectronTransport();
\end{lstlisting} 

If \(\delta\) electron transport is disabled, 
the number of electrons returned by \texttt{GetCluster} is 
the number of ``primary'' ionisation electrons, 
\textit{i.\,e.}~the photo-electrons and Auger electrons. 
Their kinetic energies and locations are accessible 
through the function \texttt{GetElectron}.

If \(\delta\) electron transport is enabled (default setting), 
the function \texttt{GetElectron} returns the 
locations of the ``conduction'' electrons as calculated by the 
internal \(\delta\) transport algorithm of Heed. 
Since this method does not provide the energy and direction of the 
secondary electrons, the corresponding parameters in 
\texttt{GetElectron} are not meaningful in this case. 

\subsection{Photon Transport}

Heed can also be used for the simulation of x-ray photoabsorption. 
\begin{lstlisting}
void TransportPhoton(const double x0, const double y0, const double z0,
                     const double t0, const double e0,
                     const double dx0, const double dy0, const double dz0,
                     int& nel);
\end{lstlisting}
\begin{description}
\item[x0, y0, z0, t0] initial position and time of the photon
\item[e0] photon energy in eV
\item[dx0, dy0, dz0] direction of the photon
\item[nel] number of photoelectrons and Auger-electrons produced 
           in the photon conversion
\end{description}

\section{SRIM}
SRIM\footnote{Stopping and Range of Ions in Matter, \href{www.srim.org}{www.srim.org}} is a program for simulating the energy loss of ions in matter. 
It produces tables of stopping powers, range and straggling parameters that 
can subsequently be imported in Garfield using the class \texttt{TrackSrim}. 
The function
\begin{lstlisting}
bool ReadFile(const std::string& file)
\end{lstlisting}
returns \texttt{true} if the SRIM output file was read successfully.
The SRIM file contains the following data
\begin{itemize}
\item
a list of kinetic energies at which losses and straggling have been computed;
\item
average energy lost per unit distance via electromagnetic processes, for each energy;
\item
average energy lost per unit distance via nuclear processes, for each energy;
\item
projected path length, for each energy;
\item
longitudinal straggling, for each energy;
\item
transverse straggling, for each energy.
\end{itemize}
These can be visualized using the functions
\begin{lstlisting}
void PlotEnergyLoss();
void PlotRange();
void PlotStraggling();
\end{lstlisting}
and printed out using the function \texttt{TrackSrim::Print()}.
In addition to these tables, the file also contains the mass and charge of 
the projectile, and the density of the target medium.
These properties are also imported and stored by \texttt{TrackSrim} 
when reading in the file. Unlike in case of Heed, the particle therefore 
does not need to be specified by the user. 
In addition, the following parameters need to be supplied ``by hand'' by the 
user.
\begin{itemize}
  \item
  the work function (in eV) of the medium (\texttt{TrackSrim::SetWorkFunction}),
  \item
  the Fano factor of the medium (\texttt{TrackSrim::SetFanoFactor}),
  \item
  the atomic number $Z$ and mass number $A$ of the medium (\texttt{TrackSrim::SetAtomicMassNumbers}).
\end{itemize}
Note that in the formulae used by \texttt{TrackSrim}, $Z$ and $A$ only enter in the 
form of the ratio $Z/A$. For gas mixtures, these numbers should therefore be chosen 
such that the average $\left<Z/A\right>$ of the mixture is reproduced. 

\texttt{TrackSrim} tries to generate individual tracks which statistically 
reproduce the average quantities calculated by SRIM.
Starting with the energy specified by the user, it iteratively
\begin{itemize}
\item
computes (by interpolating in the tables)  
the electromagnetic and nuclear energy loss per unit length at the 
current particle energy,
\item
calculates a step with a length over which the particle will 
produce on average a certain number of electrons,
\item
updates the trajectory based on the longitudinal and transverse scatter 
at the current particle energy,
\item 
calculates a randomised actual electromagnetic energy loss over the step 
and updates the particle energy.
\end{itemize}
This is repeated until the particle has no energy left or leaves the geometry.
The model for randomising the energy loss over a step can be set using 
the function 
\begin{lstlisting}
void SetModel(const int m);
\end{lstlisting}
\begin{description}
\item[m] fluctuation model to be used (Table~\ref{Tab:SrimFluctuationModels}); the default setting is model 4.
\end{description}
\begin{table}
  \centering
  \begin{tabular}{l l}
    \toprule
    Model & Description \\
    \midrule
    0 & No fluctuations \\
    1 & Untruncated Landau distribution \\
    2 & Vavilov distribution (provided the kinematic parameters are within the range of applicability, \\
      & otherwise fluctuations are disabled) \\
    3 & Gaussian distribution \\
    4 & Combination of Landau, Vavilov and Gaussian models, \\
      & each applied in their alleged domain of applicability \\
    \bottomrule
  \end{tabular}
  \caption{Fluctuation models in SRIM.}
  \label{Tab:SrimFluctuationModels}
\end{table}
Transverse and longitudinal straggling can be switchen on or off using
\begin{lstlisting}
void EnableTransverseStraggling();
void DisableTransverseStraggling();
void EnableLongitudinalStraggling();
void DisableLongitudinalStraggling();
\end{lstlisting}
If energy loss fluctuations are used, longitudinal straggling should be disabled.
By default, transverse straggling is switched on and longitudinal straggling 
is switched off. 

SRIM is aimed at low energy nuclear particles which deposit large numbers of electrons in a medium. 
The grouping of electrons to a cluster is therefore somewhat arbitrary. 
By default, \texttt{TrackSrim} will adjust the step size such that 
there are on average 100 clusters on the track.
If the user specifies a target cluster size, using
\begin{lstlisting}
void SetTargetClusterSize(const int n);
\end{lstlisting}
the step size will be chosen such that a cluster comprises on average 
\texttt{n} electrons. Alternatively, if the user specifies a maximum 
number of clusters, using
\begin{lstlisting}
void SetClustersMaximum(const int n);
\end{lstlisting}
the step size will be chosen such that on average there are 
\texttt{n / 2} clusters on the track. 
