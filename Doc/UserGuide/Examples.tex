\section{Examples}

Section~\ref{Sec:ExampleTube} discusses the 
calculation of transport parameters with Magboltz, 
the use of analytic field calculation techniques, 
``macroscopic'' simulation of electron and ion drift lines, 
and the calculation of induced signals. 
 
Microscopic transport of electrons and 
the use of finite element field maps are dealt with in 
Sec.~\ref{Sec:ExampleGem}. 

Sample macros and further examples can be found on the webpage 
(\href{http://garfieldpp.web.cern.ch/garfieldpp/}{cern.ch/garfieldpp}). 
 
\subsection{Drift Tube}\label{Sec:ExampleTube}

\subsubsection{Gas Table}
First, we prepare a table of transport parameters 
(drift velocity, diffusion coefficients, Townsend coefficient,
and attachment coefficient) as a function 
of the electric field \(\mathbf{E}\)  
(and, in general, also the magnetic field \(\mathbf{B}\) 
as well as the angle between \(\mathbf{E}\) and \(\mathbf{B}\)).
In this example, we use a gas mixture of 93\% argon and 7\% 
carbon dioxide at a pressure of 3~atm and room temperature.
\begin{lstlisting}
MediumMagboltz* gas = new MediumMagboltz();
gas->SetComposition("ar", 93., "co2", 7.);
// Set temperature [K] and pressure [Torr].
gas->SetPressure(3 * 760.);
gas->SetTemperature(293.15);
\end{lstlisting} 
We also have to specify the number of electric fields to be 
included in the table and the electric field range to be covered. 
Here we use 20 field points between 100~V/cm and 100~kV/cm 
with logarithmic spacing. 
\begin{lstlisting}
gas->SetFieldGrid(100., 100.e3, 20, true);
\end{lstlisting}
Now we run Magboltz to generate the gas table for this grid. 
As input parameter we have to specify the number of collisions 
(in multiples of \(10^{7}\)) over which the electron is traced 
by Magboltz.
\begin{lstlisting}
const int ncoll = 10;
const bool verbose = true;
gas->GenerateGasTable(ncoll, verbose);
\end{lstlisting}
This calculation will take a while, don't panic. 
After the calculation is finished, we save the gas table to a 
file for later use.
\begin{lstlisting}
gas->WriteGasFile("ar_93_co2_7.gas");
\end{lstlisting}

\subsubsection{Electric Field}
For calculating the electric field inside the tube, 
we use the class \texttt{ComponentAnalyticField} which can handle 
(two-dimensional) arrangements of wires, planes and tubes.
\begin{lstlisting}
ComponentAnalyticField* cmp = new ComponentAnalyticField();
\end{lstlisting} 
The \texttt{Component} requires a description of the 
geometry, that is a list of volumes and associated media.
\begin{lstlisting}
// Wire radius [cm]
const double rWire = 25.e-4;
// Outer radius of the tube [cm]
const double rTube = 1.46;
// Half-length of the tube [cm]
const double lTube = 10.;
GeometrySimple* geo = new GeometrySimple();
// Make a tube 
// (centered at the origin, inner radius: rWire, outer radius: rTube).
SolidTube* tube = new SolidTube(0., 0., 0., rWire, rTube, lTube);
// Add the solid to the geometry, together with the medium inside.
geo->AddSolid(tube, gas);
// Pass a pointer to the geometry class to the component.
cmp->SetGeometry(geo); 
\end{lstlisting}
Next we setup the electric field.
\begin{lstlisting}
// Voltages
const double vWire = 3270.;
const double vTube =    0.;
// Add the wire in the center.
cmp->AddWire(0., 0., 2 * rWire, vWire, "s");
// Add the tube.
cmp->AddTube(rTube, vTube, 0, "t");
\end{lstlisting}
We want to calculate the signal induced on the wire. 
Using 
\begin{lstlisting}
cmp->AddReadout("s");
\end{lstlisting}
we tell the \texttt{Component} to prepare the solution for the weighting field 
of the wire (which we have given the label ``s'' before).
 
Finally we assemble a \texttt{Sensor} object which acts as an 
interface to the transport classes discussed below.
\begin{lstlisting}
Sensor* sensor = new Sensor();
// Calculate the electric field using the Component object cmp.
sensor->AddComponent(cmp);
// Request signal calculation for the electrode named "s", 
// using the weighting field provided by the Component object cmp.
sensor->AddElectrode(cmp, "s"); 
\end{lstlisting}

In this (not very realistic) example, we want to calculate only the 
electron signal. We set the time interval within which the 
signal is recorded by the sensor to 2~ns, with a binning of 0.02~ns. 
\begin{lstlisting}
const double tMin = 0.;
const double tMax = 2.;
const double tStep = 0.02;
const int nTimeBins = int((tMax - tMin) / tStep);
sensor->SetTimeWindow(0., tStep, nTimeBins);
\end{lstlisting}

\subsubsection{Avalanche}
For simulating the electron avalanche we use the class \texttt{AvalancheMC} 
which uses the previously computed tables of transport parameters to 
calculate drift lines and multiplication. 
\begin{lstlisting}
AvalancheMC* aval = new AvalancheMC();
aval->SetSensor(sensor);
// Switch on signal calculation.
aval->EnableSignalCalculation();
// Do the drift line calculation in time steps of 50 ps.
aval->SetTimeSteps(0.05);
// Starting position [cm] and time [ns] of the initial electron.
// The electron is started at 100 micron above the wire.
const double x0 = 0.;
const double y0 = rWire + 100.e-4;
const double z0 = 0.;
const double t0 = 0.;
// Simulate an avalanche.
aval->AvalancheElectron(x0, y0, z0, t0);
\end{lstlisting}

Using the class \texttt{ViewSignal}, we plot the current induced on 
the wire by the avalanche simulated in the previous step.
\begin{lstlisting}
ViewSignal* signalView = new ViewSignal();
signalView->SetSensor(sensor);
signalView->PlotSignal("s");
\end{lstlisting}

\subsection{GEM}\label{Sec:ExampleGem}

\subsubsection{Field Map}

The initialisation of \texttt{ComponentAnsys123} consists of 
\begin{itemize}
  \item
  loading the mesh (\texttt{ELIST.lis}, \texttt{NLIST.lis}), 
  the list of nodal solutions (\texttt{PRNSOL.lis}), and the 
  material properties (\texttt{MPLIST.lis});
  \item
  specifying the length unit of the values given in the 
  \texttt{.LIS} files;
  \item
  setting the appropriate periodicities/symmetries.
\end{itemize}
\begin{lstlisting}
ComponentAnsys123* fm = new ComponentAnsys123();
// Load the field map.
fm->Initialise("ELIST.lis", "NLIST.lis", "MPLIST.lis", "PRNSOL.lis", "mm");
// Set the periodicities
fm->EnableMirrorPeriodicityX();
fm->EnableMirorPeriodicityY();
// Print some information about the cell dimensions.
fm->PrintRange();
\end{lstlisting}

Next we create a \texttt{Sensor} and add the field map 
component to it
\begin{lstlisting}
Sensor* sensor = new Sensor();
sensor->AddComponent(fm);
\end{lstlisting}

\subsubsection{Gas}

We use a gas mixture of 80\% argon and 20\% CO\(_{2}\).
\begin{lstlisting}
MediumMagboltz* gas = new MediumMagboltz();
gas->SetComposition("ar", 80., "co2", 20.);
// Set temperature [K] and pressure [Torr].
gas->SetTemperature(293.15);
gas->SetPressure(760.);
\end{lstlisting}

In this example, we calculate electron avalanches using 
``microscopic'' Monte Carlo simulation, based directly on the 
electron-atom/molecule cross-sections in the Magboltz database. 
 
\begin{lstlisting}
gas->SetMaxElectronEnergy(200.);
const bool verbose = true;
gas->Initialise(verbose);
\end{lstlisting}

In order to track a particle through the detector we have to 
tell \texttt{ComponentAnsys123} which field map material corresponds 
to which \texttt{Medium}.
\begin{lstlisting}
const int nMaterials = fm->GetNumberOfMaterials();
for (int i = 0; i < nMaterials; ++i) {
  const double eps = fm->GetPermmittivity(i);
  if (fabs(eps - 1.) < 1.e-3) fm->SetMedium(i, gas);
}
// Print a list of the field map materials (for information).
fm->PrintMaterials();
\end{lstlisting}

\subsubsection{Avalanche}

Microscopic tracking is handled by the class 
\texttt{AvalancheMicroscopic}.
\begin{lstlisting}
AvalancheMicroscopic* aval = new AvalancheMicroscopic();
aval->SetSensor(aval);
\end{lstlisting}
We are now ready to track an electron through the GEM.  
\begin{lstlisting}
// Initial position [cm] and starting time [ns]
double x0 = 0., y0 = 0., z0 = 0.02;
double t0 = 0.;
// Initial energy [eV]
double e0 = 0.1;
// Initial direction 
// In case of a null vector, the initial direction is randomized.
double dx0 = 0., dy0 = 0., dz0 = 0.;
// Calculate an electron avalanche.
aval->AvalancheElectron(x0, y0, 0, t0, e0, dx0, dy0, dz0);
\end{lstlisting}
